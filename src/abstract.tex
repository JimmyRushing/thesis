
\chapter*{Abstract}
\addcontentsline{toc}{chapter}{Abstract}

%% JF: I think the first two sentences can be combined, something like:
%% Organic semiconductors have the potential to replace inorganic materials as their production is less expensive and can be done at scale under ambient conditions. 
Semiconducting materials made from carbon-based molecules are potential replacements for inorganic semiconductors, but with lower costs of processing.
Devices made from organic semiconductors can be produced at scale by inkjet printing and roll-to-roll manufacturing of these molecules in solution or melt phases.  
The efficiency of these organic devices is dependent on the structure of the active layer, so controlling the morphology of organic molecules through self-assembly during manufacturing is a key challenge to realizing their utility.
Molecular self-assembly depends on the chemical structure of the molecules, how key moieties interact with each other and with any solvent present, and the thermodynamic paths that are sampled during processing.
Computer simulations of molecular self-assembly can predict the structure and properties of candidate systems, and can improve the amount of information gained from more expensive trials performed in a wet lab when used to guide and explain experiments.
Here we focus on the prediction of charge mobility in organic semiconducting materials,
which requires a sequence of modeling calculations spanning many orders of magnitude across both time and space.
We describe an open-source `pipeline' of calculations that serves as a virtual laboratory for the screening of organic semiconductors for their charge transport properties.
We describe work on \texttt{Planckton}, a software package for managing molecular simulations of organic semiconductors, and \texttt{MorphCT},
a package for managing kinetic Monte Carlo simulations, the modularization and testing of which improves their transparency, usability, reproducibility, and extensibility.
We measure improvements to \texttt{Planckton} and \texttt{MorphCT} by using them to study two organic molecules of interest in the photovoltaics field.
In the first case study, of semiconducting polymer \gls{p3ht}, 
we validate qualitative trends of charge mobility against prior work from both simulation and experiment.
In the second case we predict the morphology and charge transport of the semiconducting macromolecule 
\gls{itic}.
We find that our work modularizing \texttt{Planckton} improves the pace at which simulations can be iteratively tested.
We validate the electronic structure predictions made by \texttt{pySCF} against those previously made by the more restrictively-licensed \texttt{orca} package.
We measure specific features of local structure that contribute to large-scale mobility trends in P3HT and describe predictions of charge transport in ITIC.
In sum we improve the software ecosystem for reproducibly predicting charge mobility in organic semiconductors.

%%% Local Variables: 
%%% mode: latex
%%% TeX-master: "BSUmain"
%%% End: 
