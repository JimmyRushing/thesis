
\chapter*{Abstract}
\addcontentsline{toc}{chapter}{Abstract}

Carbon-based (organic) semiconductors have distinct physical properties that diverge from more traditionally
utilized metalloid semiconductors like Silicon and Germanium.
Self-assembly in these materials makes easily scalable ink-printing and roll-to-roll manufacturing techniques
possible. 
Because of their unique material properties, organic semiconductors can provide unique functionalities 
for electronic device design.
However, molecular engineering for a particular device design
requires optimizing across an array of parameters that include but are
not limited to the following: processability, ionization potential, electron affinity, permitivitty. 
Wetlab screening of materials across
this vast parameter space is prohibitive of time, cost, and creativity. 
In this work, we outline a computational, open-source, simulation pipeline that can serve as a 
virtual laboratory for the screening of organic semiconducting materials. 
Through the development of Planckton and MorphCT, convenience packages for facilitating molecular dynamics
simulations and kinetic Monte Carlo simulations respectively, we attempt to grate the learning curve
associated with organic materials simualtion. 

Organic photovoltaics (OPVs), as the name suggests, are organic semiconductors that are used in the design of
devices that turn light into electricity or vice versa.  
We test the pipeline on
two of the most well researched OPV materials for organic solar cell design and develop methods for deploying
the pipeline with open-source software engineering priniciples in mind.


Putting the chemist names for P3HT and ITIC so i can refernce later. Maybe this is an SI thing

ITIC
2,2'-[[6,6,12,12-tetrakis(4-hexylphenyl)-6,12-dihydro-dithieno[2,3-d:2',3'-d']-s-indaceno[1,2-b:5,6-b']dithiophene-2,8-diyl]bis[methylidyne(3-oxo-1H-indene-2,1(3H)-diylidene)]]bis[propanedinitrile]

P3HT poly(3-hexylthiophene)




%%% Local Variables: 
%%% mode: latex
%%% TeX-master: "BSUmain"
%%% End: 
