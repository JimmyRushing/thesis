
\chapter*{Abstract}
\addcontentsline{toc}{chapter}{Abstract}

Organic photovoltaics are an exploded \ej{Exploded?} class of advance materials that can provide unique functionalities.
All conjugated molecules can conduct charge. 
Engineering molucules for use in organic solar cells, FET's, etc \ej{polish w/o etc}
requires selecting for optimal metrics across and infinite array of tuning paramereter that effect a materials
processability, ionization potentials, electron affinity, permitivitty et cetera \ej{etc consistent?}. 
Screening materials across
such a vast parameter space is prohibitive of time, cost, and creativity. Advances in these materials evolve
from advance theoretical considerations of how a particular parameter tuning could enhance over all performace
to synthesing the new molecule to wetlab experements to confirm the results. This paradigm has successfully
\ej{Currently thinking we revisit abstract after everything else polished.}
driven OPVs to 17 PCE over the last 30 years. However, this paradigm is restrictive both creatively and
pedagogically. This approach requires a level of reductionism that is not necessary for the contribution to a
field of interest. It is my (wildly speculative) assertion that this evolutionary advancement isn't the only
way the technology advnances and that the next revolutionary advancement could come from turning the paradigm
on its head. That is to say, the experiments beget morphologies with desireable properties which begets
theoretical breakthoughs in understanding of the physical phenomena under investigation. Advancements in
computation, simulation, and software engineering could provide the playground for this. In silico (or in
whatever quantum computers are made of) could provide the aforementioned access to (virtual) laboraories that
are distributed equitably. This shift could not only provide researchers with virually limitless 
laboratories that run themselves once initiated, but perhaps more importanly could provide the labratories
with orders of magnitude more reseachers drawn from a substantially larger talent pool. It is too that end
that we implement TRUE molecular simulations.    


%%% Local Variables: 
%%% mode: latex
%%% TeX-master: "BSUmain"
%%% End: 
