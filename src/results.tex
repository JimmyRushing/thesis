\chapter{Results}
\label{results}

\section{Summary}

We present our results in four parts. 

In \autoref{qccresults}, we use MorphCT to test the performance of PySCF at the level of the chromophore.
Two experiments are reported. In the first, we calculate the frontier molecular orbital
energies for fused ring oligomers of increasing length. We compare the results of our experiment to the
results of the same experiment done with more rigerous DFT methods. In the second, we test the 
performance of the PySCF dimer calcualtions outlined in \autoref{methods}. 
We take two simple chromophores, two thiophene rings, in 4300 different oreintations and calculate the
electronic coupling between them. We do so to test if our dimer calculation correlates sensibly to the angle
and distance between chromophores. These experiments are broadly meant to confirm that we have integrated
PySCF into MorphCT properly, and that the quantites produced comport with the physics of these systems.  

In \autoref{mobility}, we deploy MorphCT on three benchmark P3HT morphologies to obtain charge mobility
and compare the results to previously reported values for these morphologies. This section is meant to
validate the current implementation of MorphCT against a previous implementation. 

In \autoref{transport}, we test the sensitivity of our MorphCT calculated mobilites to dcut, chromophore
reorganization energy, KMC temperature, and choice of charge carrier lifetimes for MSD analysis. Using a
benchmark P3HT morphology, holding all other parameters constant and sweeping accross relevant values of these
values these provides context for how to treat these parameters in future investigations. 

In \autoref{itic}, we present the results of an investigation of ITIC from molecular structure to MD predicted
morphology to KMC predicted charge mobility. We use this oppurtunity to explore the TRUEness of MorphCT as
well as the comprehensive workflow.  

\section{Quantum Chemical Accuracy}

\label{qccresults}

As outlined in \autoref{qccmethods}, QCC is used in two distinct way in our workflow. 
First it is used to estimate free energy
difference between individual chromophores as the difference in their HOMO (or LUMO) energy levels.
Secondly, it is used to estimate the electronic overlap ($T_{ij}$) with the dimer splitting method which involves
calculating the HOMO(or LUMO) of the dimer formed by the two chromophpres. We present the results from two
exploratory experiments meant to evaluate pySCF's suitablitity for performing these duties:
 (1) we compare frontier molecular energies of given by pySCF to those
given by more rigerous DFT and (2) we compare pySCF's dimer calculations to a study that used ORCA.  

\ej{The following paragraph is the beginnings of a mini methods section for these results. So, make subsections inside the results for each of these two experiments, and explain in more detail any models and methods you're using (in subsubsections!) that are relevant. }

\subsection{Experiment 1}

\subsubsection{Experiment 1 Methods}

At the level of a single chromophore, we calculate the HOMO-LUMO gap for fused-ring
oligomers of increasing length. 
The difference between the HOMO and LUMO energy levels, the HOMO-LUMO gap, is an approximation of the amount
of energy necessary to promote an electron to a higher energy level.  
Fused-ring geometries are of particlular interest for accepter
molecules, as discussed for FREAs in the introduction. 
The fused thiohpenes in this experinemt represent a generic FREA core, whose frontier molecular oribitals are
the landing cites for a charge propigating through an acceptor material. 

To recreate these experiments, using Mbuild \cite{Klein2016}, oligomers composed of 4-8 planar fused thiophene rings
were intialized and saved to a GSD file. The GSD files were fed into MorphCT which uses pySCF to quantify the
frontier orbital energy levels \autoref{qccmethods}.

\subsubsection{Experiment 1 Results}

Our values for the HOMO-LUMO gap are plotted by oligomer length in \autoref{fig:fused}. We compare these
results to those carried out with more rigerous DFT methods \cite{Arago2010}. Our HOMO-LUMO gap ranges between
$7.27eV$ and $6.34eV$. The values found using DFT were between $5.26eV$ and $4.92eV$. 

As INDO methods are known to overestimate DFT values by a factor of 2-3 \cite{Gorelsky2001}, we find that our
implementation adequately recreated these results.

Furthermore, it is known that there is a near linear
relationship between HOMO-LUMO gap and oligomer length. We find that our use of pySCF (MINDO/3) replicates
this trend and this is clear from \autoref{fig:fused}.

\begin{figure}
  \center
  \includegraphics[width = .8\textwidth]{figures/fused-ring-figure.png}
  \caption{HOMO-LUMO gap obtained using MorphCT for fused-ring molecules of oligmer length $4-8$.}
  \label{fig:fused}
\end{figure}

\subsection{Experiment 2}

\subsubsection{Experiment 2 Methods}

At the level of the chromophore pair, we explore the effect of angle and distance between two non-bonded
thiophene rings on the calculated $T_{ij}$ between them.
Using Mbuild, a reference thiophene is placed at the origin in the xy-plane
with the y-axis running through the sulfur atom. 
A second thiophene is instantiated as in \autoref{TIplots}(a)
and allowed a combination of moves wherein it is translated
in the positive z and x direction by up to 5 angstroms 
and rotated up to 90 degrees about its y-axis. Center-to-center distances less
than 3 angstroms are disallowed as distances shorter than this are considered
unphysical.

The resulting chemestries were saved to a GSD and the $T_{ij}$s quantified with MorphCT. 

\subsubsection{Experiment 2 Results}

The $T_{ij}$ resulting from these 4300 unique orientations are shown in \autoref{TIplots}(b). 
The figure shows that, as expected, a decrease in 
center-to-center distance results in more orbital overlap and thus an increase in $T_{ij}$. 
Also observable in the figure is that
a rotation about the x-axis orients the sulfer downward, resulting in a smaller sulfer-to-sulfler distance 
and a greater $T_{ij}$. Beyond these trends, the absolute value of these calculations match closely those
calculated using DFT \cite{Lan2008}, where realistic distances between thiophene rings in lamaeler P3HT
crystals between $3.8\AA$ and $4 \AA$ gives $T_{ij}$ values between $0.07eV$ and $0.1$. 

In a similar work using a random forest machine learning to predict $T_{ij}$
between thiophenes based on 9 input features, the authors found that the
features of most importance for predicting $T_{ij}$ was bonded vs non bonded,
center to center distance and rotation about the y-axis. \cite{Jankowski2019c}

\begin{figure}[]
\centering
\begin{subfigure}{.4\textwidth}
    \centering
    \includegraphics[width=\textwidth]{figures/thiophene-axes.png}
    \caption{Two nonbonded thiophenes}
\end{subfigure}%
\\
\begin{subfigure}{.8\textwidth}
    \centering
    \includegraphics[width=\textwidth]{figures/transfer_integral_plot.png}
    \caption{$T_{ij}$ between thiophene rings}
\end{subfigure}
    \caption{The orientation in figure (a) shows thiophenes 5 angstroms apart with 0 pitch. The $T_{ij}$ for
    this would appear at the origin of figure (b)}
\label{TIplots}
\end{figure}

Graduating these pairwise energetics to charge characteristics on the scale of MD simulations
requires the use of an iterative algorithm. For this we employ KMC simulations.

\section{Charge Transport Validation}
\label{mobility}
\begin{table}[ht]
    \caption{Mobility $cm^{2}/Vs$} % title of Table
\centering % used for centering table
\begin{tabular}{c c c c} % centered columns (4 columns)
\hline\hline %inserts double horizontal lines
Software & Amorphous & Semi-crystalline & Crystalline \\ [0.5ex] % inserts table
%heading
\hline % inserts single horizontal line
    ORCA & $(1.085 \pm 0.006)\cdot 10^{-1}$ & $(0.156 \pm 0.003)\cdot 10^{-1}$ & $(1.23 \pm 0.01)\cdot 10^{-1}$ \\ % inserting body of the table
PySCF & $(1.8074 \pm 0.0001)\cdot 10^{-3}$ & $(1.4751 \pm 0.0001)\cdot 10^{-3}$ & $(2.061 \pm 0.001)\cdot 10^{-1}$  \\ [1ex] % [1ex] adds vertical space
\hline %inserts single line
\end{tabular}
\label{table:nonlin} % is used to refer this table in the text
\end{table}

Having explored the performance of PySCF on the level of the molecule, we graduate these computations to the
macroscopic scale with MorphCT to obtain charge mobility. As outlined in \autoref{methods}, a predecssor to
this work has provided benchmark morphologies and the associated mobilites for these purposes. 

The results in \ref{table:nonlin} show the charge mobility reported in the prior work, which used ORCA for
the QCC calculation, and the charge mobility obtained using using the current workflow, which uses PySCF. 
The previous work found that charge mobility is highest for the crystalline morphology, followed by the
amorphous, and finally the semicrystalline. This seemingly anomolous behavoir can be explained. While the
semicrystalline morphology has more ordered high speed highways of transport, the anisotropic movement
inhibits average displacement. This anisotropy is

Our work replicates this trend. However, the absolute value of charge mobility is only an order of magnitude
match for the crystalline morphology. 

\section{Charge Transport Sensitivity Analysis}
\label{transport}

The sensitivity of the algorithm to various parameter was performed on the benchmark crystalline P3HT
morphology.

\subsection{Neighbor Cutoff (dcut)}
\label{dcutresults}
Voronoi analysis allows for the computationally efficient partitioning of space into
polyhedra chromophore cells. In this context, chromophore $i$ is the neighbor of chromophore $j$ if they
share a boundary with one another.  
This phenomena is illustrated in figure \ref{fig:dcut} 

\begin{table}[ht]
\caption{dcut sensitivity}
\centering % used for centering table
\begin{tabular}{c c c c c c c c} % centered columns (4 columns)
\hline\hline %inserts double horizontal lines
dcut & 4 & 6 & 8 & 10 & 12 & 14 & 16 \\ [0.5ex] % inserts table
%heading
\hline  % inserts single horizontal line
pairs & 318 & 22000 & 49000 & 96000 & 113026 & 113315 & 113315 \\ [1ex]% inserting body of the table
$\mu_{0}$ $(cm^{2}/Vs)$ & $-2.17 \cdot 10^{-6}$ & $6.13 \cdot 10^{-4}$ & .01 & .17 & .17 & .22 & .22 \\ [1ex] % [1ex] adds vertical space
%$\mu_{0}$ Error & $-1.75*10^{-6}$ & 6 & 8 & 10 & 12 & 14 & 16 \\
\hline %inserts single line
\end{tabular}
\label{table:dcut-sense} % is used to refer this table in the text
\end{table}

The figure shows the effect of cutoff distance on value of
calculated mobility. There is a diminishing return on computation around dcut 10. 

With each chromophore pair requiring a relatively costly QCC, after narrowing down the chromophore pairs with voronoi, it
could be
computationally preferable to calculate the distances between all pairs and remove neighbors more than dcut
apart. We can see in table \ref{table:dcut-sense}, that with dcut = 10 we get comparable mobilities to the
dcut 12 simulation with $10^5$ less
pairs, which could suggest that beyond dcut 10 there is a diminishing returns on mobility prediction with the
additional chromophores. However, we found for the materials currently under investigation, PySCF is speedy enough such that intrucing dcut adds more ambiguity
into the workflow than is necessary given that the average time per QCC dimer calculation is .036s for P3HT and
1.2s for ITIC. Furthermore, with MorphCT acting on static atomistic oreintations, these calculations 
are only necessarily performed once per morphology. With that, our workflow defaults this cutoff distance to half
the length of the simulation box, rendering it effectively moot. 

If, in the future, more computationally expensive methods are incorperated into the QCC step, or chromophores in other
organic materials are a heavier lift PySCF, it could be benificial to reintroduce this cutoff distance. The optimal dcut will vary depending on the material and before doing large sweeping analysis on a new
material its at a discount to do some preliminary analysis to determine an apporiate value. We tested the
sensitivity of the algorithm to the value (dcut). 

\subsection{Reorganization energy}

Marcus's nonadiabatic electron transfer theory allows us to model charge transfer as two
intersecting parabolic potential energy surfaces. In this work, the parabalas represent the potential energy surface
of the dimer created by a pair of chomophores with a charge injected on either chromophore. In this context, 
reorganization energy, $\lambda_{ij}$, constitutes the energy required to distort the dimer's equilibrium geometry with a
charge on chromophore $i$ into the dimers equillibrium geometry with charge on chromophore $j$.
Reorganization energy consist of the energy change associated with the destortion of the dimers geometry,
and the distortion of the surrounding medium in responce the movement of the charge. It can be written as
follows:
\begin{align}
    \lambda_{total} = \lambda_{internal} + \lambda_{external}.
\end{align} 
$\lambda = 0.3eV$ is chosen to be the default reorganization energy ($\lambda_{internal} = 0.1eV$
and $\lambda_{external} = .02eV$) as others have done with P3HT \cite{jones2017} and
a flourene-triphenylamine copolymer, TFB \cite{Gali2017}. 

\begin{figure}[]
\centering
\begin{subfigure}{\textwidth}
    \textbf{(A)} \\
    \centering
    \includegraphics[width=.8\textwidth]{figures/reorg-log-yaxis.png}
    \newline
\end{subfigure}%
\\
\begin{subfigure}{.5\textwidth}
    \textbf{(B)} $\lambda = 100$
    \centering
    \includegraphics[width=\textwidth]{figures/donor_hopping_rate_clusters_reorg100.png}
\end{subfigure}%
\begin{subfigure}{.5\textwidth}
    \textbf{(C)} $\lambda = 800$
    \centering
    \includegraphics[width=\textwidth]{figures/donor_hopping_rate_clusters_reorg800.png}
\end{subfigure}
    \caption[short]{Figure (A) shows how KMC simulated mobility values scale with the perscribed
    reorganization energy values, $\lambda$. Figures (B) and (C) show the distrubtion of hopping rates with
    $\lambda = 100eV$ and $\lambda = 800eV$ respectively. These show how $\lambda$ effects the in the Marcus
    rate accros the morphology and thus the charge mobiility.}
\label{reorg}
\end{figure}

Reorganization energy has a strong effect on charge mobility in organic
molecules[REFS]. In our workflow, reorganization energy is set as an attribute
to the chomophore object. It is defaulted to $~300meV$ as for all chromophore
objects.

To test the sensitivity of the algorithm to this value we ran 8
simulations with chromophores assigned reorganization energies of 100-800meV. The results, shown in
\autoref{reorg}, are expected from the inspection of \autoref{marcus}. Because these simulation were run on
the same morphology, the variation in distributions of $k_{ij}$ values, shown in \autoref{reorg}(B)(C), is
solely due to the choice of $\lambda$. The cumulative outcome of this is the exponetial decay in mobility as
$\lambda$ is increased.

\subsection{Temperature}

Another parameter of interest in the Marcus theory mention above is temperature. To test the sensitivity to
temperature, 15 KMC simulations from 100K to 800k were run on the benchmark P3HT crystalline morphology. It
is clear from the results in figure \ref{TEMP}(A) that the mobilities trend upward with temperature. With relatively
weak electronic coupling ($T_{ij}$) between chromophores, electron transfer proceeds nonadiabatically
\cite{clarke2010}. With this weak coupling, the temperature in the Gibbs free energy of activation term
dominates the effect temperature has hop rate.

An unanticipated result is that increasing the temperature of the KMC
simulation also increases the wall time that the simulations required to
complete. As an illustration of why this is the case, the distribution of hop
rates is plotted for 100K and 800K in figure \ref{TEMP}(C)(D). With the distribution of hop rates skewed
drastically higher at 800K, each charge carrier will experience orders of
magnitude more hops during its specified lifetime. 

If extremely cold
temperatures are to be investigates in the future, thought should probably be given to
the decreasing contribution to statistical significance from the addition of
any single charge carrier. 



\begin{figure}[]
\centering
\begin{subfigure}{.5\textwidth}
    \textbf{(A)}
    \centering
    \includegraphics[width=\textwidth]{figures/temp.png}
    \newline
\end{subfigure}%
\begin{subfigure}{.5\textwidth}
    \textbf{(B)}
    \centering
    \includegraphics[width=\textwidth]{figures/temp_simtime_plot.png}
    \newline
\end{subfigure}
\begin{subfigure}{.5\textwidth}
    \textbf{(C) 100K}
    \centering
    \includegraphics[width=\textwidth]{figures/donor_hopping_rate_clusters_temp100.png}
\end{subfigure}%
\begin{subfigure}{.5\textwidth}
    \textbf{(D) 800K}
    \centering
    \includegraphics[width=\textwidth]{figures/donor_hopping_rate_clusters_temp800.png}
\end{subfigure}
\caption[short]{A beautiful, well written caption}
\label{TEMP}
\end{figure}

\subsection{MSD (lifetimes)}

The KMC algorithm allows an explicit calculation of the MSD accross a large number of 
particles in the system. Repeating along relevant time scales for 
charge transfer, the slope of this relationship
can be estimated and related to to the 3D diffusion coeffiecient as discussed in the methods section of this
work. Using the Einstien relation (2.3), 
the groundwork for which Einstein derived in his doctoral dissertaion, finally the zero-field
mobility can be obtained. 

It is critical that we not include the ballistic transport timescale in the approximation of the limit
if the slope as time goes to infinity \cite{Maginn2018}. Estimating an upperbound for lifetimes such that
we can estimate the slope of the MSD as time goes to infinity can be messy. In real systems, free chrage
carrier lifetime is subject to a complex interplay between geminate recombination, non-geminate recombination,
charge trapping, temperature, and charge density, whose dynamics play out accros a picosecond to microsecond
timescales and vary widly form material to material as well as from microstrtucture to microstrucute for a
given material \cite{Laquai2015}.

For this work, the primary strategy was to avoid the ballistic region while exploring lifetimes
that are achievable computationally. For example, in an attempt to simulate out to the physical limit, a
simulation a microsecond($10^{-6}s$) lifetime resulted in a single hole hopping for 9 wall time hours.
It is advised that a preanalysis to determine where a given morphology's MSD will convergeb be performed so as
to never simulate past that point. Doing so is superflous and could introduce unessacery noise into the data. 

In the previous work on these P3HT morphologies, 7 liftimes were chosen and a linear regression was performed
to estimate the slope of the MSD. The current work found that the mobility can be apporpriately reproduced
with an appropriate choice of 2 lifetimes beyond the ballistic transport timescale. It was found that with the 
squared displacement averaged over 1000 holes at $10^{-9}s$ and again $10^{-10}s$ the slope of the MSD can be
commensurety reproduced with 1000's less individual holes having to be simulated. The results in the accuracy
section above were aquired with these 2 chosen lifetimes. I am currently running accross 7 lifetimes on the
crystalline morph. 

\begin{figure}
  \center
  \includegraphics[width=0.8\linewidth]{figures/lifetime.png} 
    \caption{The results of running 5 KMC simulations with the first lifetimes as described in the text. It
    can be seen that below the the ballistic timescale (roughly $10^{-10}$), the resulting mobility is affected}
  \label{lifetime}
\end{figure}

With the slope between MSD at the first lifetime and at the second lifetime proving an estimate of the slope
of the MSD, 6 simulations we run with progressiveily shorter first lifetimes. That is, the second lifetime was
set to $10^{-8}s$ for all 5 simulations and the first lifetime was set to $10^{-9}s$, $10^{-10}s$,
$10^{-11}s$, $10^{-12}s$, and $10^{-13}s$ respectively. The results are plotted in figure
\ref{lifetime}. As expected, as the first lifetime progresses in the ballistic transport timescale, which this work
estimates around $10^{-10}s$, the resulting mobility increases. If the starting lifetime is even shorter the
workflow breaks down becuase as can be seen from the hop rate distribution in figure \ref{TEMP}(D), even at
extreme temperatures, holes need more time that that to hop even once. 

Interestingly, the algorithm seems to be quite robust against choice of lifetimes. As can be seen in the
figure, order of magnitude differences in lifetime choices results in less than 2X difference in the resulting
mobility. This further suggests that going lifetime crazy is a waste of computation.  

\section{ITIC}
\label{itic}

To explore the extensibility of the workflow on another benchmark OPV material, using planckton-flow, a 1000
molecule morphology of ITIC was equillibrated over a 10e-7 step MD simulation at room temperature. From the results of the MD
simulation, the last frame of the atomic trajectories is taken to represent an accruate equalibrium geometry
of ITIC. The purpose of this work isnt to justify the efficacy of this type of molecular simulation but it has
been used to affect elsewhere[REFS].

To apply the hopping model to this atomistic morphology requires the
delineation of segments within the morphology upon which charges can delocalize along the within a the HOMO(or in
the case of LUMO for the acceptor). It should be noted that acceptor in this context doesn't refer to
chomophore $j$ as such.  
This nonadiabatic marcus theory apply to high coulpling for the 7 molecule thing is justified by matty becuase
it works. In ITIC no bonding between chromos insures the weak coupling limit where hopping model is more
easliy justified.
The LUMO of ITIC delocalizes along the backbone of the molecule, with
negligible electron density in the side chains. This has been quantified using DFT and well visualized at the
level of the molecule by Han et al \cite{Han2019}. We have visualized this at the nanometer scale in figure \ref{ITIC} using the openly
available visualization tool OVITO \cite{Stukowski2010a}. This makes the the backbone of this kind of small molecule
material the obvious choice. In more polymeric materials like P3HT, the hopping model has received further
justification elsewhere. 

\begin{figure}[]
\centering
\begin{subfigure}{.5\textwidth}
    \includegraphics[width=\textwidth]{figures/ITIC-blackedout-unwrapped-allatom.png}
\end{subfigure}%
\begin{subfigure}{.5\textwidth}
    \includegraphics[width=\textwidth]{figures/ITIC-blackedout-unwrapped.png}
\end{subfigure}
    \caption[short]{1000 molecule ITIC morphology. Right: Segments known to participate in frontier
    molecular orbitals are in color and side chains in black.}
\label{ITIC}
\end{figure}



The the reported
experimental electron mobility of ITIC varies depending on how it was processed and how it was measured. Time-of-flight electron mobilities on the order of $10^{-4}$ \cite{Mica2018} and field effect mobilities on the order of
$10^{-2}$ \cite{Park2018} have been reported. 

A single molecule ITIC has 186 atoms, with the backbone consisting of 70 atoms. We deployed two different
approaches to the delineation of chomophores within the ITIC morphology. In addition the the simulation of hopping
from backbone to backbone we ran simulations with the whole molecule prescibed as chromophores. This
necessarily requires more heavy lifting from PySCF but is trivially easy from an indexing perspective. Similar to the results of the DCUT
investigation, this implemenation of PySCF combined with clever pickling of system objects at various stages
of the workflow suggest that the laborious delineation of chromophores either via manual perscription or
by clever use of smarts matching is a fools errand. We found that, while the 70 atom chomophores took 1.2s
per dimer while the whole molecule dimer calculations with 186 atoms per chomophore to on average 3.3s.
Comprable mobilities of $(1.019 \pm 0.001)\cdot 10^{-3}$ or the backbone chromophores and 
$(1.275 \pm 0.001)\cdot 10^{-3}$ [rerunning this because it got over written] for the whole molecule. 

For small molecule materials this suggests forgoing labor intinsive indexing of chromophores.

As discussed in the sensitivit analysis, our mobility calcualtions are relatively sensitive to the choice of
reorganization energy. For ITIC, $\lambda_{internal}$ has been well investigated and is widely reported as
$~0.15eV$. The external reorganization is harder to estimate. We take $\lambda_{total}=0.3eV$ as we did for
P3HT. With the fused
backbone resulting in a higher internal contribution and the lack of long range order resulting in a lower
external contribution. 

%%% Local Variables: 
%%% mode: latex
%%% TeX-master: "BSUmain"
%%% End: 
