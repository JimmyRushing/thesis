\chapter{Conclusion}
\label{conclusion}

As others have argued \cite{Evans2016}\cite{Gali2017}\cite{Jones2017}, the kind of workflow outlined in this thesis
can allow for cheaper and more expansive screening of OPV materials across varied chemistries and processing
conditions. 

We have shown that it is theoretically possible and practically
achievable to simulate OPVs. 
Through the publication
of workflows, code development, and data we have outlined a \glsxtrlong{true}
pipeline through which any researcher can quantify material properties in
organic semiconductors. We have validated and performed sensitivity analysis on our pipeline using benchmark
\gls{p3ht} morphologies. Through our study on \gls{itic},  have also seen that the pipeline is readily
extensible to other \gls{opvs}.

Throughout this work, the most critical resource has been collaboration with
other developers. This collaboration took place in real time with active
developers and asynchronously through the trails of bread crumbs left behind
by predecessors in the form of documentation, searchable public communication,
and tutorials. This is unsurprising, because learners that are new to an area
of research, or to an application programming interface, will often experience
similar pitfalls. Open source software development provides a scaffolding
around which we can take note of these pitfalls and actively work to smooth the 
path for the next
researcher to expand the boundaries of the research further.

Through this lens, we have seen that the frayed edges, the pure
scientific theory and the pure data/computer science, of scientific software
development can entangle an aspiring researcher. For this particular project,
that meant learning quantum theory and data science simultaneously. We have
attempted to tie these ends together by creating tutorials that expose a new
user to the \texttt{MorphCT} code base and to the scientific theory
underlying it. Maintaining these tutorials as part of the code base allows
future researchers to modify and expand on them. 

As for the algorithm employed with \texttt{MorphCT}, we have learned that
\texttt{PySCF} is quantitatively and computationally sufficient for providing
\gls{qcc} across large atomistic morphologies. We found that the efficacy of
\text{PySCF} allows for the simplification of our algorithms for neighbor
listing and chromophore delineation.  





%Others have had success intruducing dynamic disorder via nudging the QQC calculated TI value at every
%iteration of the KMC algorithm by a number drawn randomly from a Gaussian Distribtion \cite{Gali2017}.

%Dynamic disorder does effect charge mobility (CT in high mobility conjugated polymers) with every pico second
%may have a 10-20\% fluxuation in TI. Need to find the paper but using the dynamic dimer methor for TI calc
%could be a way to introduce dynamic disorder. Or it could be done as mentioned before in another paper where
%they use two different methods one with a guassians fluxuation and sime ohther fancy method. 

%In our kmc algorithm we could incorperate the rates of other events like recombination.

%Mention Grits

%The computational bottle neck of simulation charge mobility is also the most theoretically challenging.
%Estimating the TI for each pair of chromophores. A future improvement to MorphCT will be accelerating this step
%with Machine learning techniques. Musil et al. have shown that ML could produce a 90 \% decrease in
%computational effort \cite{Musil2018}

%Something to do is allow morphct as to figure out what to do with donor accpeter mixtures. It would be pretty
%easy to run asyncronously on the donor/acceptor materials and compare. this could be a measure of morphology.
%If you run pure donor and pure acceptor and you got mobities, if on of those drops in the mixed morph more
%that the other you probably dont have a nice domain for BHJ. could potential screen mixtures and see average
%diffusing length before intersection with interface

%I hope one day a submission to a quantum computer could solve with extreme precicission the shrodinger
%equation for an entire morphology. Quantum algorithms are most advanced in this space and we await the race
%to large qubit computers. 

%I havent looked into it but I think reorganization energry should scale in out algorythm with crystallinity
%iasfaras we can measure crystallinity. the external reorganization energy of the material ,at least for p3ht
%was 200, and thats the lions share of overall reorganization energy which we have all ready shown to
%monitonically scale with the calculated mobilioty in this algorith. But external reorganization energy is
%related to the distortion of the surrounding medium. if the surrounding medium is shaped like a net if will
%certainly distort less than a disordered medium. Maybe this all ready talked about but energy does scale with
%molucaler size and rigidity \cite{McMahon2010a}

%5In the
%same way  that we appropriated GAFF for use in OPVs, we hope morphCT will be similarly canabalized by researcher in the other fields, like the ones oulined above. \ej{Polish later}

%in the context of decorrelation time and testing the workflow accross independent sample of equillibrium
%microstates instead of just one at a time. 

%Can validate further to see if we can recreate the tie chains for polydisperse work.

%%% Local Variables: 
%%% mode: latex
%%% TeX-master: "BSUmain"
%%% End: 
