\chapter{Conclusion}
\label{conclusion}

As others have argued \cite{Evans2016}[this is a long list], the kind of workflow outlined in this thesis
can allow for cheaper and more expansive screening of OPV materials accross varied chemistries and processing
conditions. Beyond providing proof that it is theoretically possible to simulate these materials, we seek to
show that it is also practically attainable to do so. To that end,
particular effort is given to developing TRUE simulation tools. 

Others have had success intruducing dynamic disorder via nudging the QQC calculated TI value at every
iteration of the KMC algorithm by a number drawn randomly from a Gaussian Distribtion \cite{Gali2017}.

Dynamic disorder does effect charge mobility (CT in high mobility conjugated polymers) with every pico second
may have a 10-20\% fluxuation in TI. Need to find the paper but using the dynamic dimer methor for TI calc
could be a way to introduce dynamic disorder. Or it could be done as mentioned before in another paper where
they use two different methods one with a guassians fluxuation and sime ohther fancy method. 

In our kmc algorithm we could incorperate the rates of other events like recombination.

Mention Grits

The computational bottle neck of simulation charge mobility is also the most theoretically challenging.
Estimating the TI for each pair of chromophores. A future improvement to MorphCT will be accelerating this step
with Machine learning techniques. Musil et al. have shown that ML could produce a 90 \% decrease in
computational effort \cite{Musil2018}

Something to do is allow morphct as to figure out what to do with donor accpeter mixtures. It would be pretty
easy to run asyncronously on the donor/acceptor materials and compare. this could be a measure of morphology.
If you run pure donor and pure acceptor and you got mobities, if on of those drops in the mixed morph more
that the other you probably dont have a nice domain for BHJ.

I hope one day a submission to a quantum computer could solve with extreme precicission the shrodinger
equation for an entire morphology. Quantum algorithms are most advanced in this space and we await the race
to large qubit computers. 

I havent looked into it but I think reorganization energry should scale in out algorythm with crystallinity
iasfaras we can measure crystallinity. the external reorganization energy of the material ,at least for p3ht
was 200, and thats the lions share of overall reorganization energy which we have all ready shown to
monitonically scale with the calculated mobilioty in this algorith. But external reorganization energy is
related to the distortion of the surrounding medium. if the surrounding medium is shaped like a net if will
certainly distort less than a disordered medium. Maybe this all ready talked about but energy does scale with
molucaler size and rigidity \cite{McMahon2010a}

In the
same way  that we appropriated GAFF for use in OPVs, we hope morphCT will be similarly canabalized by researcher in the other fields, like the ones oulined above. \ej{Polish later}

in the context of decorrelation time and testing the workflow accross independent sample of equillibrium
microstates instead of just one at a time. 

%%% Local Variables: 
%%% mode: latex
%%% TeX-master: "BSUmain"
%%% End: 
