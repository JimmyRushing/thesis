\chapter{Introduction} 
\indent Organic semiconductors have a niche to fill across the landscape of electronic
device manufacturing due to the flexibility, processability, and cost of manufacturing. They have all ready
achieved commercial viability in the OFET, OLED space NEED REFS. They are of interest to a vast array of
advance material engineering. These materials can
provide stretchability, self-healing, and biodegradability for next generation medical devices
\cite{Brutting2006}.  Morphologies were obtained via in house software (see github for planckton-flow).  The
fuel for the simulation engines mentioned above(plankton-flow/hoomd) utilized in this work is the generalized
Amber force field (GAFF)\cite{Wang2004a} The Amber forcefield was designed for use in modeling protein and
nucleic acid systems.  Serendipitously, the generalized Amber forcefield has parameters for organic molecules
comprised of H,C,N,O, and P and produces accurate simulations of organic molecules for use in OPVs.  

BHJ OSC work in 4 steps . photoabsorbtions, exciton diffusion, CT , Charge diffusion. \citet{Fusella2019}.
Engineering the active layer of an OSC from organic materials presents distinct challenges and advantages at
all four stages. Achieving effecient photoabsorbtion in the OPV active layer requires designing donor and
acceptor molecules with high absorbtion in the most abundant region of the solar spectrum. An example
of an advantage of excitonic photoabsorbtion is that, absortion takes place in a narrow peak of roughly 300nm.
This allows for the ability to tune the active layer material to absorb radiation in just above or below the
visible spectrum (into the NIR or UV spectrum respectiviely). This, along with solution processability, 
could allow for the production of windows that passively draw current. Semi-transparent OSCs have all ready
reached 11 \% efficeincy. \cite{Brabec2020}

\newline \indent Our work
focuses on stage four. It has been show that the hole/electron of both donor and acceptor material are
relavant to overall device performance. \cite{Wang2019e}.  Something about if electron mobililty it too high
in acceptor vs the hole mobility in the donor. It gets clogged up and you get nongeminate recombination loss
Also, imbalanced charge-carrier mobility can lead to space-charge build up in the low mobilty material that
can screen the built in field.  \cite{Bartelt2015}. shown by space charge limited current experiments
\cite{Small2013}.

%%% Local Variables: 
%%% mode: latex
%%% TeX-master: "BSUmain"
%%:set textwidth=80
% End: 
