\chapter{Introduction} 
Organic semiconductors have a niche to fill across the landscape of electronic
device manufacturing due to the flexibility, processability, and cost of manufacturing. They have all ready
achieved commercial viability in the OFET, OLED space NEED REFS. They are of interest to a vast array of
advance material engineering. These materials can
provide stretchability, self-healing, and biodegradability for next generation medical devices
\cite{Brutting2006}.
SECTION
In solar cell design, the proximate motivation for this research is the wide range of novel way that OPVs
can assist us in harnessing the power of the sun. 
An example of an advantage of the excitonic photoabsorbtion that takes plave in OPVs is that, absortion takes place in a narrow peak of roughly 300nm
wide.
This allows for the ability to tune the active layer material to absorb radiation in just above or below the
visible spectrum (into the NIR or UV spectrum respectiviely). This, along with solution processability, 
could allow for the production of windows that passively draw current. Semi-transparent OSCs have all ready
reached 11 \% efficeincy. \cite{Brabec2020}

Taking for granted that electronic devices that utilize self-assembling organic materials are an exciting
space, we find it highly desirable to have a pipeline through which we can simulate the equilibrium
chemistries of these materials and characterize material properties 

[LIST OF OTHER NEAT THINGS] In the area of OPVs, as recently as 1986, Ching W. Wang
showed that the right donor and acceptor molecules, processed under the right conditions, will self-ssemble
into what is termed Bulk Heterojunction (BHJ) microstructure. And an OPV active layer with this microstucture
can overcome the
coloumb force(augmented by the low dielectric constant in the denominator of coloumbs law) between the excited
electron and the hole it left behind in the valence band. An exciton can diffuse roughly 10nm(Clarke) before relaxing
to its ground state and photoemitting the energy. The interlocking phases of donor accepter molecules, insure
that an exciton will intersect with the boundary between accepter and donor domains, where the slight offset in
energy levels between the donor and accepter molecules creates a charge transfer state wherein it is more
energetically favorable for the donor to undergo electron transfer with an adjacent acceptor than it is to radiatively decay to its ground state.

[essentially trying to tell the story of the BHJ story, and the story of the NFA to illustrate that it takes
incredible theoretical breakthoughs and years of wet lab testing. The reason for painting this narrative ito
motivate how nice it would be to test these theories with computers first to accelerate the
advancement]
Reaching 17 \% effeciency in these devices required rigerous theory and time consuming chemical synthesis
that unfolded over the course of three decades. 
With the immensity of parameter space involved in the design of BHJ material, it is desirable that a modular
open-source pipeline exhist for simulating the morphology of candidate molecules as well as simulating the 
associated charge charecteristics. We note that will motivate zero-field mobility as a relvant quantity in BHJ
design, it could likely be informative accross all of the aforementioned applications. 


BHJ OSC work in 4 steps . photoabsorbtions, exciton diffusion, CT , Charge diffusion. \citet{Fusella2019}.
Engineering the active layer of an OSC from organic materials presents distinct challenges and advantages at
all four stages. 
\indent Our work is expressely related to stage four. It has been show that the hole/electron mobility of both donor and acceptor material are
relavant to overall device performance. \cite{Wang2019e}. If electron mobililty it too high
in acceptor vs the hole mobility in the donor, it gets clogged up and you get nongeminate recombination loss
Also, imbalanced charge-carrier mobility can lead to space-charge build up in the low mobilty material that
can screen the built in field.  \cite{Bartelt2015}. shown by space charge limited current experiments
\cite{Small2013}.
Morphologies were obtained via in house software (see github for planckton-flow).  The
fuel for the simulation engines mentioned above(plankton-flow/hoomd) utilized in this work is the generalized
Amber force field (GAFF)\cite{Wang2004a} The Amber forcefield was designed for use in modeling protein and
nucleic acid systems.  Serendipitously, the generalized Amber forcefield has parameters for organic molecules
comprised of H,C,N,O, and P and produces accurate simulations of organic molecules for use in OPVs. In the
same way we hope morphCT will be similarly canabalized by researcher in the other fields, like the ones oulined above. 


\indent Jones and Jankowski provided just such a pipeline. Version one of MorphCT was capable of
connecting qualitativley the morphological features produced by MD simulations. However, it was determined
that in the interest of reprducabitilty, it was critical to ``containerize'' this pipeline. Until as recently as
the past few years it was common place to not publish code with the results of computational works. Containers
are virtual machines that contain all the dependencies, configurations, code and data necessary to reproduce
results. \cite{Cito2016a}

THIS SENTECE HAS NO HOME Researchers have to
balance the optimiztion of electron structures of candidate donor/acceptor materials against the miscibility
of two candidate molecules as well as the resultant morphology across the thermodynamic landscape of
variuos solution processes which ultimately govern the Jsc and FF of at the device level \cite{Zhu2020a}. 
%%% Local Variables: 
%%% mode: latex
%%% TeX-master: "BSUmain"
%%:set textwidth=80
% End: 
