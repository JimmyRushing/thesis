

DOI: 10.1039/D1CP00674F
    These guys have there own morphCT at https://github.com/tobiaskoch1988/charge_transport_package

DOI DOI: 10.1002/aic.16760

    Machine learning predictions . Evans paper

    Tested five techniques for using machine learning to predict mobility and random forests were the most succesful

    15 percent efficiency is all you need for one day pay back times .

    fed into the maching was 1. bonded or non bonded
                                2. distance between thiothenen rings center of mass
                                3. angles between thiothene rings
                                4. distance between sulfurs in thiothen rings
                                5. x y and z coordinat distances between cetners
                                6. energy difference between chomophores


DOI doi.org/10.1063/5.0047421 

    DOPING RELATED BROADENING OF DENSITY OF STATES GOVERNS INTEGER-CHARGE TRANSFER IN P3HT
    
    conjugated polymers can be engineered to have high "figure of merit" ZT=ST/(\rho*\kappa) where the demonetor 
    is resistivity vs thermal conductivity. thermal conductivity is small in conjugated polymers and resistivity can be 
    significantly doped away. 

    they say that ICT and FCT are in competition and that by adding a p-dopant with electron affinity higher that the
    ionization energy of the surrounding material(P3Ht) you can encourage ICT. I seems to me like FCT is ignored in JONES2017
    via the assumption that a chromophore is one monomer. ie you cant have the excitation delocalized over large chains. it has to hop. 
    Also should we  can we "dope" our simulations of P3HT

DOI DOI:10.1038/s41578-019-0137-9

CHARGE TRANSFER ELECTRONIC STATES IN ORGANIC SOLAR CELLS

    electron-hole pairs are frenkel-type excitons. bulk heterojunctions are required becuase to dissasociate these we need a
    donor acceptor interface where a CT state can be generated. This is characerized by the electron sittin in the lumo of the 
    acceptor and the hole sitting in the homo of the donor

    they say a three state model is needed to fully describe the interation between the 1.) the electronic state if the exciton 
    2.) the electronic state of the CT state and 3.) the electronic state of the ground state. but a two state approximation can
    be used to ignore the interation between the absorbed electron state. that is to say a a charge tranfer state and a ground state 
    have weak vibronic enteraction with the the local electron state. almost like the rate limiting step approximation

    thermally induced and light induced electron transfer processes are connected by the mulliken-hush model.
    
    the electronic coupling between between ground and CT states is related to the adiabatic transition dipole moment. (i
    take this to mean the dipole moment at the adaiabatic limit. meaning that the maximuim coupling at which electron transfer 
    happens adiabatically. the potential energy surfaces intersect. )

    energy distributions of CT states generally follw a gaussiain distribution do to the relation of electron-vibration interaction. 

    urbach energy is a measure of disorder and takkes contributions from static and dynamic disorder
    
    the Voc in OSCs is lower than traditional cells. this is attricbuted to the fact that the CT-state has a lower energy
    than the optical gap and there for lower energy than the photo induced electron. voltage is also lose to non-radiative
    decay to ground states. voltage loss can be describes in terms of the quantum efficiency of electroluminescence EQE.  this
    value is proportional to the CT decay rates. all that is to say that voltage loss increase with a decrease in CT-state energy.



DOI: Liu, Y. (2021). Spectroscopic study of charge-transfer states in organic semiconductors (Order No. 28320730). Available from ProQuest Dissertations & Theses Global. (2526355653). Retrieved from https://libproxy.boisestate.edu/login?url=https://www.proquest.com/dissertations-theses/spectroscopic-study-charge-transfer-states/docview/2526355653/se-2?accountid=964

    based on the Shockley-Quiesser theoretical framework, an ideal solar cell has no non-radiative reconbination. only radiative which is
    unavoidable. They say that a great solar cell is a great LED because it has small non-radiative recombination. 
    basically, energy in is enrgy out in one form or the other.

    something new that this work is doing is considering the local excitons(specifically thier lifetimes) 
    influence on the photolumninesent yield of the CT state. Could be reading this worng but this is a consideration explicitally
    ignores in the the two state theory

    they state as does the work above that the simple bilayer organic photovaoltaics is doomed because the excitons cant diffuse far enough
    (they diffuse up to 10nm probably impossible to make a 5nm thin flim)to reach a dissasotiating interface. 

    fullerene acceptors are hard to tune. 

    they say non-symmetrical acceptors could lead to highly intermixed morphology which would require novel morphological control to tune 
    exciton splittting and charge transtport. 

    over all energy loss is defined as the difference bewtween E_{s1}, the energy of the low band gap exciton and the free carriers E_{carriers}
    or, \Delta E = E_{s1} - E_{carriers} .


    They are studying 4TIC which has a more eleictron rich donor core , 4T, which they expect to give rise to
    stronger intramolecular charge transfer with IC. they say that 4TIC extends the absorption into the 
    nIR (near infared ) wavelengths compared to og ITIC.


DOI: 	Phys. Chem. Chem. Phys., 2015,17, 28451-28462
        
"HOT OR COLD :HOW DO CHARGE TRANSFER STATES AT DONOR-ACCEPTOR INTERFACE OF AN ORAGNIC SOLAR CELL DISSACSOCIATE."
        
DOI: 10.1021/cr900271s

"CHARGE PHOTOGENERATION IN ORGANIC SOLAR CELLS"
    
    Onsanger theory - Onsanger first described geminate recombination. the model caculates the probablity
    of a CT state  disassociating in terms of the dielectric constant and the temerature of the system and the 
    applied electric field.

    to begin, he defines a coulumb attraction radius. this is the the distance at whcih the coulumb attraction
    between hole and electron is equal to the thermal energy. this is the beginning of the proposed "hot" electron
    in whcih the the exess thermal energy from the photoabsorbtion pushes the electron and hole apart at a 
    thermalization length "a". 

    if this lenght "a" is larger than the coulumb attraction radius, the charges are considered delocalized.

    if this length is within the capture radius, then there is a probabilty P(E) which is a functiuon of the
        applied electric field. this is the probablilty of escape the coulumb attraction and become free charges.

    With this model, this leaves a probability of 1-P(E) of the CT state decayin to the ground state. 

    the dielectric constant is in the denomenator of the capture radius. in inorganic conductors the dielectric 
    constant is large such that very few electron-holes stick together because the capture radius is very small. 

    if there is no applied electric field, the P(E) just scales proportionally with the distance "a". 

    Braun modified Onsanger to consider the finite lifetimes of CT states. he took the dissoctiation 
    to free charges to be a reversible process. that is the CT state may be dissociating and reassociating
    many times before it fully dissociates or decays to the ground state. 

    Tachiya noted that Onsanger works for moderate to low electron mobility materials. because it messes up the 
    mean free path. (toolong) . it also doesnt account for reoganization energies. it is also hard to uncover the
    "a" and and the capture radius. 

    from wikipedia- a singlet state an excited electron has oppisite spin from the one left in the ground
    state energy level  and in a triplet state, the excited electron has the same spin as the one left behind
    in the gound state energy level. 

    a fundemental requirement for dissasociation is that the the charge seperated state must be the
    lowest energy excited state in the system . if the donor ionization potential is large enough that 
    it elevates the energy of the charge seperated state above other possible states, this would cut off
    the pathway for charge transfer.

    onsangers capture radius model can inform domain sizes in BHJ . for instance if the domain size
    is smaller than the capture radius the charges cant escape each other. this could be tricky to implement,
    as the exciton diffusion length and the capture radius are often of similar magnitudes. what is agreed upom 
    is that nanomorphology strongly effects charge dissociation and recombination

	

DOI: 10.1016/B978-0-08-102284-9.00020-6

	OPV: DEVICE PHYSICS

	
    The energy potential of the light striking the earth is over thirteen time that of all known reserves of
    coal, natural gas, petroleum, and uranium combonined. Perez and Peresz 2015

    to improve on traditional solar cells clunkyness, engineers tried to use amorphous silicon, CdTe and CIGS
	(copper indium gallium diselenide). This can be processed into thin-films. These are the 
	competetors for organic thin films. They still use rare earth materials and require
	a great deal of energy to fabricate. OPV fabrictate themselves if we can fairy god mother
	them with the right thermodynmics and substrates 
	OPVS solution processability allows them to be formed on lightwieght substrated and on
	curved surfaces. The materials mentioned above require too much heat to by processed on
	lightweight substrates.And with that lightwieght we have cheaper shipping and
	installation costs.

    Because P=IV power equals current *voltage , the J-V curve is a useful metric for measuring 
	device perfomrance. Where j is I averaged over area. 

    OPV materials have lower mobilities but higher absorbtiion coefficients. the absorbtion coefficient
	is a measure of how far light can penetrate into a material. 

   in 1986 Ching Tang invented the Bulk heterojunction. because of the alignment of the LUMO of the
	donor and acceptor, the excited donor finds it more enjoyable to relax via charge transer than 
	via the radiative decay to the  ground state. for holes the alignment urges the acceptor to
	transfer a hole to the donor.

   EQE external quantum efficiciency is a fancy name for the fraction of extracted photons
	to total number of photons incident on a device
		-it can be written explixitly as the photo absorbtion efficiency times
		the exciton diffusion efficiency times the CT efficiency times the 
		charge extraction efficiency. that is the combined efficiency of
		the 4 steps of OPV charge collection. 

   IQE Ineternal quantum efficiency is a fancy term for the fraction of extracted electrons
	to the number of photons ABSORBED into the active layer of the device IQE and
	EQE are equal in the limit of effeciency. that is, where every incident photon is 
	absorbed by the active layer 
		-this is the product mentiond above without the first efficiency accounted for. 

	Determining IQE is harder becuause you need to know the absoption spectrum and this
		hard to observe experimentally becuase its a thin film and the surroundings 
		interfere with the observation. 
DOI: ?

	Controlling energy levels and Fermi level en route to fully tailored energetics in organic semiconductors


	READ THIS NEXT. it has a github for its statistics as well. 

   		fermi level of intrinstic semiconducter directly between the conduction band 
		and valence band. because fermi level is the energy at which there is a 50 percent chance 
		of having an electron with this energy. 

		in extrinstic semiconductor. in n-type thre fermi level is shifted higher towards the 
		conduction band because the probability of finding an electron in the conduction band 
		is higher than probabiktiy of finding a hole in the valence band. vice versa for 
		p-type. these notes from youtube "femi level of intrinsic and extrinsic semiconductor. 
