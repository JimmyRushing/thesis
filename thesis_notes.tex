

DOI: 10.1039/D1CP00674F
    These guys have there own morphCT at https://github.com/tobiaskoch1988/charge_transport_package

DOI DOI: 10.1002/aic.16760

    Machine learning predictions . Evans paper

    Tested five techniques for using machine learning to predict mobility and random forests were the most succesful

    15 percent efficiency is all you need for one day pay back times .

    fed into the maching was 1. bonded or non bonded
                                2. distance between thiothenen rings center of mass
                                3. angles between thiothene rings
                                4. distance between sulfurs in thiothen rings
                                5. x y and z coordinat distances between cetners
                                6. energy difference between chomophores


DOI doi.org/10.1063/5.0047421 

    DOPING RELATED BROADENING OF DENSITY OF STATES GOVERNS INTEGER-CHARGE TRANSFER IN P3HT
    
    conjugated polymers can be engineered to have high "figure of merit" ZT=ST/(\rho*\kappa) where the demonetor 
    is resistivity vs thermal conductivity. thermal conductivity is small in conjugated polymers and resistivity can be 
    significantly doped away. 

    they say that ICT and FCT are in competition and that by adding a p-dopant with electron affinity higher that the
    ionization energy of the surrounding material(P3Ht) you can encourage ICT. I seems to me like FCT is ignored in JONES2017
    via the assumption that a chromophore is one monomer. ie you cant have the excitation delocalized over large chains. it has to hop. 
    Also should we  can we "dope" our simulations of P3HT

DOI DOI:10.1038/s41578-019-0137-9

CHARGE TRANSFER ELECTRONIC STATES IN ORGANIC SOLAR CELLS

    electron-hole pairs are frenkel-type excitons. bulk heterojunctions are required becuase to dissasociate these we need a
    donor acceptor interface where a CT state can be generated. This is characerized by the electron sittin in the lumo of the 
    acceptor and the hole sitting in the homo of the donor

    they say a three state model is needed to fully describe the interation between the 1.) the electronic state if the exciton 
    2.) the electronic state of the CT state and 3.) the electronic state of the ground state. but a two state approximation can
    be used to ignore the interation between the absorbed electron state. that is to say a a charge tranfer state and a ground state 
    have weak vibronic enteraction with the the local electron state. almost like the rate limiting step approximation

    thermally induced and light induced electron transfer processes are connected by the mulliken-hush model.
    
    the electronic coupling between between ground and CT states is related to the adiabatic transition dipole moment. (i
    take this to mean the dipole moment at the adaiabatic limit. meaning that the maximuim coupling at which electron transfer 
    happens adiabatically. the potential energy surfaces intersect. )

    energy distributions of CT states generally follw a gaussiain distribution do to the relation of electron-vibration interaction. 

    urbach energy is a measure of disorder and takkes contributions from static and dynamic disorder
    
    the Voc in OSCs is lower than traditional cells. this is attricbuted to the fact that the CT-state has a lower energy
    than the optical gap and there for lower energy than the photo induced electron. voltage is also lose to non-radiative
    decay to ground states. voltage loss can be describes in terms of the quantum efficiency of electroluminescence EQE.  this
    value is proportional to the CT decay rates. all that is to say that voltage loss increase with a decrease in CT-state energy.



DOI: Liu, Y. (2021). Spectroscopic study of charge-transfer states in organic semiconductors (Order No. 28320730). Available from ProQuest Dissertations & Theses Global. (2526355653). Retrieved from https://libproxy.boisestate.edu/login?url=https://www.proquest.com/dissertations-theses/spectroscopic-study-charge-transfer-states/docview/2526355653/se-2?accountid=964

    based on the Shockley-Quiesser theoretical framework, an ideal solar cell has no non-radiative reconbination. only radiative which is
    unavoidable. They say that a great solar cell is a great LED because it has small non-radiative recombination. 
    basically, energy in is enrgy out in one form or the other.

    something new that this work is doing is considering the local excitons(specifically thier lifetimes) 
    influence on the photolumninesent yield of the CT state. Could be reading this worng but this is a consideration explicitally
    ignores in the the two state theory

    they state as does the work above that the simple bilayer organic photovaoltaics is doomed because the excitons cant diffuse far enough
    (they diffuse up to 10nm probably impossible to make a 5nm thin flim)to reach a dissasotiating interface. 

    fullerene acceptors are hard to tune. 

    they say non-symmetrical acceptors could lead to highly intermixed morphology which would require novel morphological control to tune 
    exciton splittting and charge transtport. 

    over all energy loss is defined as the difference bewtween E_{s1}, the energy of the low band gap exciton and the free carriers E_{carriers}
    or, \Delta E = E_{s1} - E_{carriers} .


    They are studying 4TIC which has a more eleictron rich donor core , 4T, which they expect to give rise to
    stronger intramolecular charge transfer with IC. they say that 4TIC extends the absorption into the 
    nIR (near infared ) wavelengths compared to og ITIC.


DOI: 	Phys. Chem. Chem. Phys., 2015,17, 28451-28462
        
"HOT OR COLD :HOW DO CHARGE TRANSFER STATES AT DONOR-ACCEPTOR INTERFACE OF AN ORAGNIC SOLAR CELL DISSACSOCIATE."
        
DOI: 10.1021/cr900271s

"CHARGE PHOTOGENERATION IN ORGANIC SOLAR CELLS"
    
    Onsanger theory - Onsanger first described geminate recombination. the model caculates the probablity
    of a CT state  disassociating in terms of the dielectric constant and the temerature of the system and the 
    applied electric field.

    to begin, he defines a coulumb attraction radius. this is the the distance at whcih the coulumb attraction
    between hole and electron is equal to the thermal energy. this is the beginning of the proposed "hot" electron
    in whcih the the exess thermal energy from the photoabsorbtion pushes the electron and hole apart at a 
    thermalization length "a". 

    if this lenght "a" is larger than the coulumb attraction radius, the charges are considered delocalized.

    if this length is within the capture radius, then there is a probabilty P(E) which is a functiuon of the
        applied electric field. this is the probablilty of escape the coulumb attraction and become free charges.

    With this model, this leaves a probability of 1-P(E) of the CT state decayin to the ground state. 

    the dielectric constant is in the denomenator of the capture radius. in inorganic conductors the dielectric 
    constant is large such that very few electron-holes stick together because the capture radius is very small. 

    if there is no applied electric field, the P(E) just scales proportionally with the distance "a". 

    Braun modified Onsanger to consider the finite lifetimes of CT states. he took the dissoctiation 
    to free charges to be a reversible process. that is the CT state may be dissociating and reassociating
    many times before it fully dissociates or decays to the ground state. 

    Tachiya noted that Onsanger works for moderate to low electron mobility materials. because it messes up the 
    mean free path. (toolong) . it also doesnt account for reoganization energies. it is also hard to uncover the
    "a" and and the capture radius. 

    from wikipedia- a singlet state an excited electron has oppisite spin from the one left in the ground
    state energy level  and in a triplet state, the excited electron has the same spin as the one left behind
    in the gound state energy level. 

    a fundemental requirement for dissasociation is that the the charge seperated state must be the
    lowest energy excited state in the system . if the donor ionization potential is large enough that 
    it elevates the energy of the charge seperated state above other possible states, this would cut off
    the pathway for charge transfer.

    onsangers capture radius model can inform domain sizes in BHJ . for instance if the domain size
    is smaller than the capture radius the charges cant escape each other. this could be tricky to implement,
    as the exciton diffusion length and the capture radius are often of similar magnitudes. what is agreed upom 
    is that nanomorphology strongly effects charge dissociation and recombination

	

DOI: 10.1016/B978-0-08-102284-9.00020-6

	OPV: DEVICE PHYSICS

	
    The energy potential of the light striking the earth is over thirteen time that of all known reserves of
    coal, natural gas, petroleum, and uranium combonined. Perez and Peresz 2015

    to improve on traditional solar cells clunkyness, engineers tried to use amorphous silicon, CdTe and CIGS
	(copper indium gallium diselenide). This can be processed into thin-films. These are the 
	competetors for organic thin films. They still use rare earth materials and require
	a great deal of energy to fabricate. OPV fabrictate themselves if we can fairy god mother
	them with the right thermodynmics and substrates 
	OPVS solution processability allows them to be formed on lightwieght substrated and on
	curved surfaces. The materials mentioned above require too much heat to by processed on
	lightweight substrates.And with that lightwieght we have cheaper shipping and
	installation costs.

    Because P=IV power equals current *voltage , the J-V curve is a useful metric for measuring 
	device perfomrance. Where j is I averaged over area. 

    OPV materials have lower mobilities but higher absorbtiion coefficients. the absorbtion coefficient
	is a measure of how far light can penetrate into a material. 

   in 1986 Ching Tang invented the Bulk heterojunction. because of the alignment of the LUMO of the
	donor and acceptor, the excited donor finds it more enjoyable to relax via charge transer than 
	via the radiative decay to the  ground state. for holes the alignment urges the acceptor to
	transfer a hole to the donor.

   EQE external quantum efficiciency is a fancy name for the fraction of extracted photons
	to total number of photons incident on a device
		-it can be written explixitly as the photo absorbtion efficiency times
		the exciton diffusion efficiency times the CT efficiency times the 
		charge extraction efficiency. that is the combined efficiency of
		the 4 steps of OPV charge collection. 

   IQE Ineternal quantum efficiency is a fancy term for the fraction of extracted electrons
	to the number of photons ABSORBED into the active layer of the device IQE and
	EQE are equal in the limit of effeciency. that is, where every incident photon is 
	absorbed by the active layer 
		-this is the product mentiond above without the first efficiency accounted for. 

	Determining IQE is harder becuause you need to know the absoption spectrum and this
		hard to observe experimentally becuase its a thin film and the surroundings 
		interfere with the observation. 
DOI: ?

	Controlling energy levels and Fermi level en route to fully tailored energetics in organic semiconductors


	READ THIS NEXT. it has a github for its statistics as well. 

   		fermi level of intrinstic semiconducter directly between the conduction band 
		and valence band. because fermi level is the energy at which there is a 50 percent chance 
		of having an electron with this energy. 

		in extrinstic semiconductor. in n-type thre fermi level is shifted higher towards the 
		conduction band because the probability of finding an electron in the conduction band 
		is higher than probabiktiy of finding a hole in the valence band. vice versa for 
		p-type. these notes from youtube "femi level of intrinsic and extrinsic semiconductor. 


DOI:doi.org/10.1016/j.orgel.2012.07.008

	KINETICS OF CHARGE TRANSFER PROCESSES IN ORGANIC SOLAR CELLS: IMPLICATIONS FOR THE
	DESIGN OF ACCEPTOR MOLECULES

		Non-fullerenes are nice becose they are more tunable. tuning the electronic states c
		can extend the Voc of the device whcih can extend the FF box of the J-V curve
		which ddescribes the power of the device. 
		
		potentially cheaper and more versitile processability

		learning how and why they improve or not on Fullerenes can give insight
		into the mechanisms governing efficeint charge transport

		At the time of this paper they propose that bad morephology, lower permitivity 
		(dielectic constant) and poor charge generation at the interface could explain
		why NFAs havent matched the performance of fullerene deriviatives. I beleive 
		NFAs out perform now. 

		at the interface, the ionization potential of the excited donor (the energy necessary
		to rip an electron away) and the electron affinity of the acceptor (the energy released
		when an electron is forced into the valence band) govern the gibbs free energy for
		driving force for charge transfer and is ultimately the Voc. 

		We approximate the difference between EA and IP as the difference between lumos of
`		donor and acceptor. they claim that this is inexact and inaccurate but that 
		there is a minimum of this lumo difference is still necessary to overcome the exciton 
		binding energy. that is \DeltaE = Exciton binding energy	

		in fullerene blends IQI can approach EQE.

		in this study the investigate PET of different accptors mixed with P3HT to show
		show that the energetics alone dont suffice to explain the kinetics of PET
		in this materials.

		steady state photoluminesnce intendity isnt a reliable quantitative reliable measure
		of over all PL intensity do to film nonuniformity and other geometric distubances.
		they look for CHANGES of an order of magnitude to indicate PET. 

		They used cycloc votammetry to determine the EA of the various acceptors. 

		with DFT an marcus hush they see the acceptors in the sweet spot of the marcus
		curve are indeed the materials with most effeiceint transfer
			
		To conclude they say to bypass this and calculated the electronic coupling.

DOI: https://doi.org/10.1002/aenm.202001864

	MATERIAL STRATEGIES TO ACCELERATE OPV TECHNOLOGY TOWRAD GW TECHNOLOgy

		-opv performance increased from 8 percent to 17 percent from 2009 to 2019
		-all efficeincys above 17 percent were recorded with NFA. 

DOI: 10.1002/solr.202000029

        EXPERIMENTALLY CALIBRATED KINETIC MONTE CARLO MODEL REPRODUCES ORAGNIC SOLAR CELL J_V CURVE

    -The losses in BHJ are (1) excitons not diffusing to the interface before they relax to the ground state
        (2) geminate recombination and (3) non-geminate recombination. 

    -they use KMC to create an accurate full device predictive J-v curve. The leap is that they dont
        need to describe the morphology in the kmc becuase the CT state life times are what matters
        from morphology and they can measure that directly from TA experiments. 


DOI: doi.org/10.1016/j.matpr.2020.05.677

        THE EFFECT OF INJECTION BARRIER ON OPEN CIRCUIT VOLTAGE OF OPV

    -An in injection barrier for an electron(hole) at the cathode (anode) is basically the energy 
    barrier from the fermi level of the cathode(anode) into the LUMO(HOMO) of the acceptor.

    -This paper tries to investigate the injection barrier theoretically, claiming it is very hard to
    measure experimentally. 

    -did read the guts but they claim 0.1eV is the limit of how large the injection barrier should be
        in order top maximize Voc. 

    -from the context of this paper it seems paramoun that the mobility for electrons in acceptors
    should be balanced by the mobility of holes in the donor. that is, you cant have your holes 
    ttransporting way faster than the electron without mucking up your Voc.
    

DOI: https://doi.org/10.1116/1.1559919

        METAL_ORGANIC INTERFACE AND CHARGE INJECTION IN ORGANIC ELECTRONIC DEVICES

    -charge injection has a large inffluence on device performace. my rabit hole today is to figure out 
    what the hell a charge injection is!

    -electrodes either inject or extract charges. why is it always about charge injection? arent we 
    concerned with charge  extraction. do a certain amoount need to be injected to create an electric
    field??? is the whole thing about how injecting contacts effect current as opposed to getting it
    from photons?

    answers here: http://depts.washington.edu/cmditr/modules/opv/physics_of_solar_cells.html

        -photo current is a reverse current. so its kinda the fight between the forward bias and
        the reverse current that is power. 

DOI: 10.1039/c5ra20085g

        ASSESMENT OF DENSITY FUNCTIONAL METHODS FOR EXCITON BINDING ENERGIES AND RELATED
        OPTOELECTRONIC PROPERTIES

    -the exciton bonding energy or E_{b}, is the energy required to rip a hole and electron apart. 

    -(From wikipedia for exciton: the wave function of a bound exciton state is an exotic atom 
        state. and is "hydrogenic", that is hyrdogen like. its a quasi particle because it has no 
        actual nulcei but it behave like a particle. What is intersting is that the exciton binding
        energy is smaller that to a hydrogen atom, but the "particle" is much bigger. this makes
        sense because hydrogens are tiny.)
    -it is unclear according to this article from 2015 wether DFT can accurately describe or 
    predict this binding enegy. 

    -this paper took 121 different types of small to medium sized molecules and and caclulated
    ,with two different DFT protocols (protocols?), the exciton binding energies. they say
    the 

    -the result is that the wB97, wB97X, wB97X-D out perform most other density functionals.

    -when a photon is absorbed it can be absorbed in a single molecular unit (frenkel exciton)
    or it can be spread over multiple monomers and they call this a intermolecular charge-transfer
    exciton. 

    -as a first step lee et al only test these functionals on frenkel excitons. 

    -directly measuring exciton bonding energies is hard but they can be obtained from 
    difference between fundemental and optical gaps. ->from google search this means if the 
    optical gap is the energy from ground state to excitated stated and the fundemental gap
    is the energy to go from ground state to free electron becuase to measure the fundemental 
    gap you put a photon and get an exciton out (roughly speaking), then the difference is just
    how much energy it took to push it from exciton to free charge. ie E_{b}

    -they examine the IP, EA, fundemental gaps, and optical gaps. SEE pdf labeled Barford_2013
    for simple diagram of these energy levels.

    -the moral of the story is all DFTs are equal but some are more equal than others
    

DOI: 10.1021/acs.jpcc.8b12261

    -this paper takes a look at what adding flourine does to ITIC
    -reasons they say that flourination helps: (1) higher backbones planarity
    (2) impacts to morphology like higher crystallinity, more pure domains, reduced domain size.
    -explanations for 2 coould be that flourine can make noncovalent interanctions (nonbonded interactions))
    with S and H for example. 
    -E_{b} lowered by .4 electron volts. RESULTS FROM DFT CALCULATIONS. 


DOI: https://doi.org/10.1021/cr050140x

        CHARGE TRANSPORT IN ORGANIC SEMICONDUCTORS

    -my hope is that this paper will give context to how we use MSD to calculate charge mobility
    -we assume that transport is purely diffusive (which i suppose is equivelent to saying its 
    zero field.) 
    -diffusion is a local displacement around an average position and drift is a displacement of 
    the average position
    - small molecules and oligomers usually processed in vacuum and polymers in solution
    
DOI: https://www.youtube.com/watch?v=YFzeMMOvhl0&t=4273s&ab_channel=Ren%C3%A9M.Williams


    -lecture on PET. 
    -If you excite the donor you have a lumo-lumo interaction.
    -if you excite the accpetor you have a homo-homo interaction.
    -in opvs you almost always excite the donor
    - this lecture contais the best visualizations of orbital interactions that I have seen
    -from wiki oxidation state is anachronistic because oxygen losing electron was the first time
    this process was quantified


DOI: thoughts from scipy conference

    -Open source matters
    -The framework created to run the mars helicopter was created right out in the open on git hub
    -the first observed black holes were obsereved using open source tools
    -need to do a liturature review that motivated revamping morph using open source tools
    -:


DOI:10.1016/j.orgel.2015.11.021
    
    ESTIMATION OF CHARGE CARRIER MOBILITY IN AMORPHOUS ORGANIC MATERIALS USING RANDOM WALK


        -these guys use a similar model as morph ct. they used 50 CPUs. even with that they found
        that this "in silico" investigations can parse candidates for opvs. but times have changed since 
        2006 in terms of large data management/open source/ gpu computing . because , much of the heavy lifting
        of these simulations requires pairwise calculations the workflows are highly parrelizable and well
        suited to gpu computing.


DOI:10.1119/1.2188962

    EINSTIEN AND BROWNIAN MOTION. 

    -Of the imfamous einstien papers published in 1905, the most cited, and possibly least heralded was the 
    publuication of his doctoral theses. The paper was "A New determination of Molecular Dimensions". He 
    vant Hoffs law and stokes law together to estimate the size of dissolved sugar molecules. On his way, He 
    derived a formula for the diffusion coefficeint of a suspended picroscopic particle. With similar assumptions, 
    in his 1905 paper on Brownian motion, he would go on to derive the Einstien equation, which relates the
    MSD of a particle to this diffusion coeeficiont. 

https://www.youtube.com/watch?v=Qg7pQ_CYaIQ&ab_channel=MITOpenCourseWare

    WAVE PARTICLE DUALITY OF MATTTER

    -de broglie won a noble prize for wavelength = planks constant divided by (mass * velocity) which was 
    essentially his PHD thesis

    - a 75 mile an hour fast ball has  a wave length 10^-34. thats not a relevant frequncy
    to a base ball. an electron has roughly a wavelegth of 2 *10^{-10} = 2 angstroms. The diameter of
    an atom is anywhere from .5 to 4 angstroms. so this electrons wave function is very relevant

    -

DOI: 10.1002@chem.200903343

    - The ionization potential value decreases with oligomer length due to the extension of
    the pi-conjugated backbone."IPs were calculated as the energy differ- ence between 
    the minimum-energy optimized structures of the radical cation and the neutral molecule." 
    this is just homo-lumo?  with 4 fused ringed thiophenes at 5.26eV to 4.92eV for 8 fused
    ring . this is pretty easily reproducable with morphCT? So i ran a quick notebook on this.
    call it fused-experiment. and the band gap is reduced by 5 percent. the above ionization potential
    is a reduction of 7 percenct. 

    -display an almost perfect linear dependence on the reciprical number of fuseed thiophenes

DOI: https://doi.org/10.1021/acs.chemrev.8b00803
    
    QUANTUM CHEMISTRY IN THE AGE OF QUANTUM COMPUTING

    - doing quantum chemistry with classical computers has been a square peg in a round hole for
    50 years

    - its critical that every thing we do is modular because when we want our algorithms to 
    combine with the results from quantum computers not be suplanted

    -Hartree-fock method or Hartree equation is what MINDO3 uses. it is an approximate solution
    to the shrodinger equation 

    -this paper is over my head but extremely relevant to informing what is the most efficeint
    path to simulating organic molecules. What if the fastest way to software that accurately 
    simulatates organic molecules is to start making software that works on quantum computers?

    -
DOI: 10.1145/2889160.2891057

    USING DOCKER CONTAINERS TO IMPROVE REPRODUCIBILITY IN SOFTWARE ENGINEERING RESEARCH

    -reproducibility is a problem. not just from peer to peer but from mentor to mentee
    Docker/ singularity containers work. the reus proved it. 
    - Docker is an open source container. containers are virtual machines that contain all the
    dependencies, configurataion,, code and data necessary to reproduce results. 
    -
    
DOI: doi.org/10.1145/2723872.2723882

    AN INTRODUCTION TO DOCKER FOR REPRODUCIBLE RESEARCH

    -one would think, computer codes should be more easilty reproducible than
    wet lab experemiments. That is, unless you have ever tried to do it. 
    -they say the back in 2015 people didnt even want to publish the code to their work?
    I mean, my code sucks but if you want to dive in.....
    -and they say that many journals dont even require it????
    -they say its not in the curriculum. 
    -two made heuristics for achieving reproducibility. 
        1.workflow software- This on is GUI heavy but requires mass adoption
        not adopted because no one is willing to pay for with public moneys
        and if companies make it, it is proprietary. 

        2.virtual machines- this approach hijacks the whole operating system. They propose
        running on the cloud. critics say that these monolithic containers will keep researchers
        from weaving together new software

    - the DevOPS strategy is basically script everything. 
    - docker images are binary files that conatain all the software is all ready there and
    configured. this addressses the dependency issue because you are reproducing the 
    exact code that the researchers produced.
    - the difference between a other VMs and docker is that the docker image shares a kernal 
    with the host machine. 
    -docker images have built in version control
    -

DOI: 10.1371/journal.pone.0177459

    SINGULARITY: SCIENTIFIC CONTAINERS FOR MOBILITY OF COMPUTE

        
    -random thought: if, when, quantum computers become critical to various industries and
    researchers, the tactics and workflows bolstered from remote computations on high performance 
    clusters will be critical . because no one will have a local quantum machine.
    -they say docker is fire but it isnt optimized for HPC's and to use it would introduce security
    risks
    -docker has no native support for GPUs
    -they have a nice spreadsheet of what singularity uses vs what docker uses
    -wow, found a typo "In essence, mobility ofcompute means being able to contain the entire software stack, from data files up through the library stack, and reliability move it from system to system." reliably . either way
    this is the title of the paper so a good definition to have around. 

DOI:10.1109/HPCC/SmartCity/DSS.2019.00362

    EXPLORING THE PERFORMANCE OF SINGULARITY FOR HIGH PERFORMANCE COMPUTING SCENARIOS

    -singularity meshes well with SLURM
    -basiacally using singualrity negligeby slows down the HPC computing but
    awards many many benifits. 

DOI: 10.1016/j.energy.2020.116921
    
    THERMOMECHANICAL CHARACTERISATIONS OF PTFE, PEEK, PEKK AS ENCAPSULATION MATERIALS FOR MEDIUM TEMPERATURE SOLAR APPLICATIONS

    - peek and pekk are also of interst to solar enegy applications for use as macroencapsulation
    materials for PCM (phase change material). I guess the heat turns it into a liquid and 
    you turn it back into a solid to extract heat . 
    -10.1016/j.rser.2013.07.028 this paper explains how thermal energy storage works. 
    dreamer thought: solar cells work worse when they get hot so use a TES system to extract heat
    and store it for later. two birds one stone. 
