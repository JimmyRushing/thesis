
Wed Jun 16 15:27:22 MDT 2021
    
       In the 10.1002/cphc.201900793 paper it says that the highest ITIC transfer integral 
       is 16Mev. this motivates finding a morphology of itic that is well packed and 
       seeing if out TIs match this. in the transfer-qcc-pluck notebook im goin go 
       have it grab chromos that are near each other and give the TI for them.
    
        I have conscripted gwen and cody to produce rdfs from /home/erjank_project/itic-project/workspace 
        to pick an educated guess of a nice morphology to begin with. that is, a non liquid or frozen one
        at least.

        the guts to achieve this are in the intramolecular notebook

Thu Jun 17 12:31:30 MDT 202

      Going to get some benzene rings from mbuild in some oreintations and get transfer integrals 
      between them. I will call the notebook transfer-qcc-P3HT and i will build the benzenes in 
      new_molecule_test notebook

      evans paper gives TI between .3 and 1 for various conditions ie bonded non-bonded, seperation
      distance etcetera

      instead of specifying the exact angles between thiophenes to compare to evans paper,
      I am going to use fill_box from mbuild to give me a sudo random orientations of two
      thiothenes in a box.

      getting reasonable numbers for thiophenes in terms of homo humo and TI. jenny going to get
      some more sceintifically shifted gsds.
      

Wed Jul  7 13:08:05 MDT 2021

      Working through jennys modified workflow for morphct. Hole mobility is negative which I beleive
      not good.

Thu Jul  8 13:06:52 MDT 2021
        
     In our working example of p3ht mobility calculator, we are getting a negative mobility. 

Sun Aug 22 19:54:41 MDT 2021
    
    I thought the code was broken because I kept getting a nan for TI which resuted in a zero TI
    but it wasnt broken. When a para meta two mer is relaxed in mbuild the meta and para have very different
    homos whcih results in a negative number under a square root in the TI calc. But We get a .11 TI for an
    unrelaxed para_meta. This is very different. a zero TI will result in no hop. Maybe we should use koopmans
    aFri Jan 21 20:23:42 MST 2022pproximation to avoid this?

Mon Sep 20 11:30:00 MDT 2021
    
    Digging into the mp going on in morphct I believe we can delete p.join() from 
    mobility_kmc. the purpose of it is two wait to execute the rest of the program
    until all the children have finished. However, it was malfunctioning and I beleive
    that using the connection pipe as Morphct the .recv() end of the pipe blocks the
    program until the .send() end of the pipe sends somthing . This will achieve 
    a what p.join() was trying to do. 

    problem trying to get it going on fry. I dont know how to change the flow. 
    I am getting permission errors. 

    I tried to add an operation like this: 
    (https://github.com/mosdef-hub/reproducibility_study/blob/c54751f88739b16eb1c61174d88e48e7382cf699/reproducibility_project/src/engines/hoomd/project.py#L38-L45)
    with a directive but I got operation function has no
    attribute .with_directive .
    
Thu Sep 23 11:24:56 MDT 2021
    I cant find any DFT or experimental numbers for PEKK mobility values. Is a time of flight o
    pure peek and pekk something that southern miss can do?
        
Fri Jan 21 20:23:42 MST 2022
    
    im going to run a script in fry that saves a pickle of the three large morphologies at at each step of the 
    process. That way any analyis I need to do I can pick it up at that pickle. 

Thu Jan 27 11:51:27 MST 2022

    I tried the see that the MSD would look like with life times at 10^-9 and 10^-6. But one carrier took 9
    hours to stop hopping. LOL. My thought was if we took a linear fit from a time well past ballistic
    transport and then a time way in the future we should get the true msd slope 
